\chapter{Darknets}

\section{What is a P2P network, what a darknet and how do they diver?}
In modernen Computernetzen nicht mehr nur klassisches Client/Server-Model; unter anderem p2p; filesharing hat viel zur entwicklung beigetragen; aber auch content-delivery entwicklung; wird vermehrt auch auf anonymitaet wertgelegt;

darknetze als p2p netze in denen nur mit vertrauten knoten kommuniziert wird; es sollten keinerlei informationen an nicht-vertraue gelangen; informationen: sowohl wahre identit"at, angebotene & abgerufene informationen; als auch allein die teilnahme am netzwerk selbst; kann als geschlossene gruppe oder friend-to-friend realisiert sein;

bei gruppe kennt/vertraut gezwungenermaßen jeder jedem und alle teilnehmer sind bekannt; routing und suchen kein wirkliches problem; kann im prinzip klassisches P2P mit zugangskontrolle zb ueber zentrale instanz oder gemeinsamen geheimniss sein;

im gegensatz zu gruppe wird bei friend-to-friend auch die teilnahme am netz selbst vor anderen (außer den direkten kontakten) geheim gehalten; hierbei sogar oft unklar von wem genau reqeust/response zu information kommt 


\section{...}
\begin{itemize}
\item    Detailed explanation of characteristics
\begin{itemize}
\item        (short) what is a P2P overlay; how they differ
\item        only connected to people one trusts
\item        no (concrete) topology information leakage
\end{itemize}
\item    Implications of those characteristics
\item    Survey of existing darknets
\begin{itemize}
\item        (historical perspective ?)
\item        mainly grouped/sorted by used test/analysing methods
\item        implemented/pure theoretical work
\end{itemize}
\item    Recap and explain used metrices and analysis methods
\end{itemize}
