\chapter{Darknets}

In this chapter the concept, the characteristics and the thereby arising problems of darknets are explained. Metrices and measurement methods for darknet evaluation are discussed. Existing darknet approaches are surveyed for the used evalutaion methods. 

\section{What is a P2P network, what a darknet and how do they diver?}

In modern computer networks not only the classic client/server architecture is used but also peer-to-peer (P2P) networks have gained importance. In this concept every client takes additionally a server-like role and distributes information to others clients, also called peers. Although probably the main driving force for its development was the use of filesharing, today P2P is adapted for communication, content delivery and multimedia streaming.

But in conventional P2P networks neither the membership in the network nor the stored or requested content is consealed. Since the demand for privacy and anonymity preserving communication channels has risen, some P2P concepts with respect to these needs have emerged. In such requirement needing environments like regieme critical or wistle blowing communitys it is essential that the membership in such a network and even more importantly the activities in it are kepts secret. This includes the untracability of both the requester and the publisher of files and the location of files in the network. Even the metadata of files like the filesize have to be protected since they may allow conclusions which content is requested. P2P networks respecting these requirements are called darknets.

Therefor darknets are P2P overlays in which participants connects and communicates only to others they have some trust relation with. Nodes do not pass to whom they are connected to. This leads to only a minimum number of trusted participants knowing of ones membership in a darknet. Reqeusts for storing, searching and receiving files are forwarded to other nodes but are modified to look as they come form one self. The same is done accordingly with responses. Thereby no node on such a forwarding chain can tell if a request originates from the node it received the request from or any node beyond it. Any data distributed in the network is chunked in same-sized parts which are addressed and distributed individually. Furthermore as they are encrypted no node storing or forwarding them knows what files they contain.

\section{Implications of darknet characteristics}

In summary the membership of a node in a darknet is only known to its trusted peers, the files stored on a node are unknown to and unreadable by this node and requests/responses can originate from either the node they are received from or any node beyond this. This high rate of protection of privacy relevant information comes with numerous difficultiy in designing and evaluating simultaneously resiliant and scalable darknets.



\section{metrices for routing (specially in darknets) evaluation}

For measuring and comparing routing algorythms several metrices exist. Not all of them a of practical use if it comes to peer-to-peer and darknet networks. A simple first aproach  is the duration a packet takes to reach its destination, often refered to as the ping to the destination. But since this depends on many factors, which even can vary over time, it is not quite comparable and thus unusable for measuring and evaluating routing algorythms.

A less varying metric is the path lenght, the amount of hops a packet has to be forwarded on until it reaches its destination. The path length, and its average and maximum in a network, is the most commonly used metric to compare routings.



\section{Available measurement methods for darknets}


\section{Survey of previous darknets}


