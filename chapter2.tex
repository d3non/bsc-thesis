\chapter{Darknets}

\section{What is a P2P network, what a darknet and how do they diver?}
(noch mit auffuehren: chunking und verschluesselung des contents; ist aber fuer routing irrelevant)

In modernen Computernetzen nicht mehr nur klassisches Client/Server-Model; unter anderem p2p; filesharing hat viel zur entwicklung beigetragen; heute viel entwicklung bei content-delivery;

darknetze als p2p netze in denen nur mit vertrauten knoten kommuniziert wird; es sollen keine informationen an nicht-vertraue gelangen; informationen: sowohl identitaet, angebotene & abgerufene informationen; als auch allein die teilnahme am netzwerk selbst; kann als geschlossene gruppe oder friend-to-friend realisiert sein;

bei gruppe kennt/vertraut gezwungenermassen jeder jedem und alle teilnehmer sind bekannt; routing und suchen kein wirkliches problem; kann im prinzip klassisches P2P mit zugangskontrolle zb ueber zentrale instanz oder gemeinsamen geheimniss sein;

im gegensatz zu gruppe wird bei friend-to-friend auch die teilnahme am netz selbst und die eigenen kontakte (moeglichst) vor anderen (ausser den direkten kontakten) geheim gehalten indem keine status/routing/topologieinfos in das protokoll einfliessen

ausserdem oft verschleierte weitergabe von reqeusts/responses;

hier betrachtung von friend-to-friend systemen

\section{Implications of darknet characteristics}

ausser netzwerktraffic zu/informationen bei nachbarn keine hinweise auf teilnahme am netzwerk; durch verschleierung des ursprungs von requests/responses so gut wie keine topologieinfos;

problematisch bei routing, da erstmal unklar welches der beste naechste hop/node ist; aber auch suche/abfrage von infromation da ggf unklar wer sie hat/wo sie ist

macht einerseits ueberwachung andererseits messung (fuer entwicklung/verbesserung) fast unmoeglich; aber zum entwickeln und verbessern muss man verhalten testen u bewerten koennen.

\section{metrices for routing (specially in darknets) evaluation}

For measuring and comparing routing algorythms several metrices exist. Not all of them a of practical use if it comes to peer-to-peer and darknet networks. A simple first aproach  is the duration a packet takes to reach its destination, often refered to as the ping to the destination. But since this depends on many factors, which even can vary over time, it is not quite comparable and thus unusable for measuring and evaluating routing algorythms.

A less varying metric is the path lenght, the amount of hops a packet has to be forwarded on until it reaches its destination. The path length, and its average and maximum in a network, is the most commonly used metric to compare routings



\section{Available measurement methods for darknets}

\section{Survey of previous darknets}

\section{outline (old)}
\begin{itemize}
\item    Detailed explanation of characteristics
\begin{itemize}
\item        (short) what is a P2P overlay; how they differ
\item        only connected to people one trusts
\item        no (concrete) topology information leakage
\end{itemize}
\item    Implications of those characteristics
\item    Survey of existing darknets
\begin{itemize}
\item        (historical perspective ?)
\item        mainly grouped/sorted by used test/analysing methods
\item        implemented/pure theoretical work
\end{itemize}
\item    Recap and explain used metrices and analysis methods
\end{itemize}
