\chapter{Darknets}

In this chapter the concept, the characteristics and the thereby arising problems of darknets are explained. Metrices and measurement methods for darknet evaluation are discussed. Existing darknet approaches are surveyed for the used evalutaion methods. 

\section{What is a P2P network, what a darknet and how do they diver?}

In modern computer networks not only the classic client/server architecture is used but also peer-to-peer (P2P) networks have gained importance. In this concept every client takes additionally a server-like role and distributes information to others clients, also called peers. Although probably the main driving force for its development was the use of filesharing, today P2P is adapted for communication, content delivery and multimedia streaming.

But in conventional P2P networks neither the membership in the network nor the stored or requested content is consealed. Since the demand for privacy and anonymity preserving communication channels has risen, some P2P concepts with respect to these needs have emerged. In such requirement needing environments like regieme critical or wistle blowing communitys it is essential that the membership in such a network and even more importantly the activities in it are kepts secret. This includes the untracability of both the requester and the publisher of files and the location of files in the network. Even the metadata of files like the filesize have to be protected since they may allow conclusions which content is requested. P2P networks respecting these requirements are called darknets.

Therefor darknets are P2P overlays in which participants connects and communicates only to others they have some trust relation with. Nodes do not pass to whom they are connected to. This leads to only a minimum number of trusted participants knowing of ones membership in a darknet. Reqeusts for storing, searching and receiving files are forwarded to other nodes but are modified to look as they come form one self. Same is done accordingly with responses, or, more general, to all messages. Thereby no node on such a chain of forwarding nodes can tell if a message originates from the node it received it from or any node beyond it. Any data distributed in the network is chunked in same-sized parts which are addressed and distributed individually. Furthermore as they are encrypted no node storing or forwarding them knows what files they contain.


\section{Implications of darknet characteristics}

In summary the membership of a node in a darknet is only known to its trusted peers, the files stored on a node are unknown to and unreadable by this node, and messages can originate from either the node they are received from or any node beyond it. This high rate of protection of privacy relevant information comes with numerous difficultiy in designing and evaluating simultaneously resiliant and scalable darknets.

Messages whose destination is not within a nodes neighbors have to be forwarded to some nodes in between. In conventional networks the next node can be choosen on the basis of topology information about the network, e.g. in form of a classical routing table or structured overlays in P2P systems. Though any topology information about the network is confidential they are not distributed and collected by the nodes and not available for deciding to which node a message is given next. This holds as well for meta topology information such as the origin of a message and therefore the direction the according node lies within.

\section{metrices for routing (specially in darknets) evaluation}

The same difficulties arise for measuring any quality in a darknet, for example for evaltuation and comparison of decissions while development. In general several metrices exist for measurement and comparison of routing algorithms. But since we utilize an abstract model of darknets we consider basic metrices as they apply to all kind of networks.

\begin{itemize}
\item The simplest metric is the \emph{path length}, or \emph{hop count}, the amount of hops a packet has to be forwarded on until it reaches its destination. It is an important factor of delays in communication and also affects the bandwidth between nodes. The shorter the choosen path is, the faster the communication is and the less the network has to be utilized. The path length, and its average and maximum in a network, are the most commonly used metrices to compare routings.

\item The ratio of amount of lost or dropped messages to the amount of sent messages is a metric for the quality and reliability of a network. 

\item
\end{itemize}

\section{Available measurement methods for darknets}


\section{Survey of previous darknets}


