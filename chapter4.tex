\chapter{Implementation}

In this chapter we describe the used simulation framework OMNeT++, its components and their purpose, the implementation of our model as a basic darknet node and its extension of possibly offline nodes.

\section{The simulation library and framework OMNeT++}

As statet above, network simulation is an both effective and efficient method to evaluate network performance, but an underrepresented one for darknets. This section will cover the used simulation framework OMNeT++.

\subsection{Network simulators and why we cose OMNeT++}

In chapter 2.5 we stated that the benefit in using simulation for routing evaluation is the reduced time and memory usage compared to emulation or a testbed. This would comes in exchange for a huge implementation effort, since a simulator, the client model and a network is needed. Fortunately simulatiors and frameworks exist already.

Besides some general purpose simulatiors and many specialized ones, there are even some for network analysation and evaluation. Widely used open source products are ns2/ns3 and OMNeT++. Additionally to saving the implementation of a simulator, using a commonly adopted framework gains a better comparability and to other results and respectability of the results by other researchers.

We chose OMNeT++ for its extendability, modular design and wide varity of already implemented telecomunication protocols shipped with the INET package for example.

\subsection{What is OMNeT++}

OMNeT++ is an simulation library and framework written in C++. It is modular and extensible and primarily used to simulate networks not only limited to telecomunication networks. Models, the basic entity, are written in C++ and can be assebled to larger models. Those compound models, up to the whole network, are described with a higher level language, the NEtwork Description language (NED).

Models, as well as the parts of a compound model, communicate with each other via messages. Messages are handed between input and output ports, in terms of OMNeT++ called gates, of models. A model can send messages on a gate and is informed if a messages arives on one. Messages are also used internaly of a model to trigger scheduled events. Gates can be connected to gates of the composing models directly so messages can be passed ``up'' and ``down'' within a compound model. Models on the same layer can be connected via channels, which can have corresponding properties to indicate the characteristics of a communication channel like the bandwith, the delay or even an error rate.

With the NED a network of modules and interconnecting channels can be composed. Along with the models of such a network, the whole simulation is configured via an .ini file.

\subsection{The simulation process}

The models code is compiled together with the simulatior. The simulator is then executed and a network is build according to the .ini file and the specified .ned modules.

After the initialisation and configuration phase of the nodes is done, the main simulation phase begins. In this phase the actual simulation process takes place. Previously scheduled events are processed in the specified order and trigger modules to exchange messages. This phase ends when no more messages are scheduled or on exceeding a maximum simulation time and is followed by the finalisation phase.

In the finalisation phase the most important task is to collect and eventually process measurements and statistics. Beside this, some modules need a proper clean up what is can be done in this phase.

\subsection{Measurement recording}

OMNeT++ provides two ways of recodring statistics. The traditional way was to calculate the values inside the modules and output them in in the finalisation phase in special vector or scalar objects. Because this required recompilation on even the simplest changes, like for example to record the minimum instead of the maximum of a value, recording by signals was introduced in OMNeT++ 4.1.

With this method the modules just have to emit a signal of a declared type with an optional value. How these singals are recorded and processed can controled via the simulation configuration.

\subsection{INET: The communication networks simulation package}

Since OMNeT++ is a general purpose network simulation framework not limited to a specific type of network, it contains no concrete network models. These are provided by additional packages. For communication networks, like the internet, this is the INET package. It contains models for a great varity of components the internet and similar networks consist off. Several wireless and wired protocoals are covered and reach from ethernet up to application protocols like HTTP.

For the darknet simulation we use the INET package to provide modular and realistic underlaying protocols. Depending on the level of realism to be achieved, the underlaying topology can be built as a single switch network or a whole internet like network with mulitple routers.

\section{Implementaion of the basic model}

As in any P2P Overlay an addressing scheme is needed for darknets as well. We chose a simple string representation for the nodes identification, the \emph{nodeID}. Other addressing schemes can be mapped to strings if needed.

The \emph{DarknetBaseNode} class implements the general behaviour of a darknet node and the basic interaction with the simulation framework components. It is responsible for sending and handling recieved messages. \emph{Selfmessages}, used in OMNeT++ for notification and scheduling events are detected and dispatched to an overwritable method. This way, subclasses can easily use and extend this notification system. A simple implementation of sending messages to other nodes is implemented, including setting the source of the message to the node, calling the method that determines the nodes the message is send to and then sending the message to all these nodes. In the easiest case, a deriaved class has only to implement the method that chooses the next nodes to gain message sending functionality.

If no routing descision is needed, or possible, a simplier methoden \emph{sendDirectMessage} is provided. It sends the message directly to the specified node. This is a huge simplification of the real world and only used in special cases like connecting an offline node that can obviously not in the connected list. The simplification at this point is valid, since in the underlying model of the simulation we concentrate only on the routing algorythm and the problem of how to contact a node whose underlay address (in the internet the IPv4 address for example) is unknown, is ignored.

The DarknetBaseNode class also basically manages the list of known & trusted peers and the list of connections to online peers. The first lists limits to which other nodes connections are allowed. The second one keeps track of those who are online, that is who are available to forwarded messages to, along with the address in the underlaying network.

The second key property of darknets, the anonymized forwarding of messages for other nodes, is simply implemented as well. The source ID of messages that have to be forwarded are set to the node ones. The nodeID from which the message was received is saved along with the \emph{request ID}. This request ID is used to determin to which node a response of a request is to be send to. This whole mechanism is only a simplistic implementation of the corresponding darknet characteristic and can be extended or overwritten with a more advanced algorythm by derived classes.

As explained in the model description, we simplified the different reqeust and response types in different P2P networks possible to only one common request and an accoring response type. If further differentiation of the message types is needed, they can be implemented by extending the message class. How a node acts up on reception of such a message type can be defined in the \emph{handleIncomingMessage} method.

\section{Trust relation and network generation}

Unlike as in a standard P2P overlay and modeled by existing P2P simulation frameworks as OverSim, a darknet node connects only to specific nodes. It is not possible to specify the (maximum) amount of nodes in the network and simply spawn or delete them on demand. Every node has to have a set of its trusted peers from the beginning.

To handle even very large social graphs of several thousands of nodes we developed a python script to generate the network composition in a .ned file and the node and simulation configuration in an .ini file. The script takes the network graph as file, starting with the nodeID and followed by all its trusted peers seperated by a blank. A directed symetric graph is assumed, in which each directed edge between two nodes has a reversly directed edge specifying them mutiality of trust between those nodes. However this is not enforced and complexer scenarios with not necessarily mutual trust relations can be modeled.

The network is set up with just a single router all nodes are connected to. To simulate complexer underlay network topologies, the script can be modified accordingly. Thereby it would be easily possible to simulate for example a split of a subset of the network by just putting some nodes on a different router and removing the channel between both routers during simulation.

\section{Churn based lifetime model}

If a node goes offline, this has to be detected by its peers. Therefor, an acknowledge mechanism is needed. Basically a acknowledge message (short: ACK) is send back on receieving a message. If such an ACK is not received within a certain amount of time, the message can be resent up to a selectable number of times. If no ACK is received after that, the peer is no longer considered the be online and connected.



\section{two simple example routing models}

\subsection{randomwalk}

\begin{itemize}
\item            with n-degree fanout
\item            very simple loop prevention
\end{itemize}

\subsection{flooding}
