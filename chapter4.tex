\chapter{Implementation}

\section{The simulation library and framework OMNeT++}

\subsection{Alternatives and why we cose OMNeT++}

\subsection{What is OMNeT++}

OMNeT++ is an simulation library and framework written in C++. It is modular and extensible and primarily used to simulate networks not only limited to telecomunication networks. The simulation 

\subsection{The simulation process}

\subsection{INET: The communication networks simulation package}

\section{Implementaion of the static model}



\section{Churn based lifetime model}
If a node goes offline it has to be detected by its peers. Therefor, an acknowledge mechanism is needed. Basically a acknowledge message (short: ACK) is send back on receieving a message. If such an ACK is not received within a certain amount of time, the message can be resent up to a selectable number of times. If no ACK is received after that, the peer is no longer considered the be online and connected.



\section{two simple example routing models}


\subsection{randomwalk}
\begin{itemize}
\item            with n-degree fanout
\item            very simple loop prevention
\end{itemize}
\subsection{flooding}

\section{notes}
kapite aus theoretischem model auf implementierung (bei kap.4.2) matchen?

probleme vor denen wir standen/ zu erfuellende requirements: beschreiben wie metriken ausgewertet werden
