\chapter{Model Description}

\section{Why we chose the simulation based approach}

As explained in Chapter 2.4, different methods to evaluate darknet systems exist, each suitable for different usecases. To analyse a routing algorythm, its performance, strengths, and weaknesses, larger scale networks are neccessary. For this, especially while dealing with changeing algorythms and parameters, the simulation approch is most appropriate since it has the best tradeoff of flexibility and scalabiliy.

\section{Node based with fixed neighbor set and add-hoc anonymity (static)}

Virtually every P2P client supports a basic set of capabilities. These are connecting to other clients, storing and recieving of information and some kind of searching. Depending on the network searching for files or other nodes can be different or the same. Naturally these reqests can or even have to be responded to.

So on an abstract level, a P2P node has to be able to connect to other nodes, make requests and send an response to a received request. This is the same for darknet nodes. The peers a darknet node connects to or accepts connections from of course limited.

Additionally a darknet client has to be abled to forward reqeuests to other nodes. In darknets the hop-anonymity is typically achieved by modifying forwarded messages as to come from one self. Responses to forwarded requests have to be modified and forwarded respectively to the origin the reqeust was received from.

This simple model is static, all nodes are online all the time and a failing node is not possible. It is very simple and scalable because only a minimal overhead is needed. But it is not very realisitc since nodes are not online all the time and can fail on one way or another.

\section{Churn model extension with bootstrapping and offline detection (dynamic)}

To take into account the posibilitie of nodes shutting down, having connection problems or failing otherwise, the model is extended by a churn based lifetime model. It gives the nodes a probabilistic lifetime, so not all nodes are online in the begining, and nodes can toggle their online/offline state during the simulation.

\section{(??) Extension of the model by example}
