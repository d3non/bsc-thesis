\chapter{Model Description}

\section{Why we chose the simulation based approach}

For evaluating distributet systems like P2P overlays, and darknets in particular, there are four different approaches.

The most theoretical one is to bulid a \emph{analyical model} and derive formulas for the relevant values from it. This is quite flexible for varying parameters and very scalable and fast. But on the other hand the derivation of formulars can be challenging and small changes in the model or algorythm can render most of the work inappropriate. With analytical models upper and/or lower bounds can be estimated, but real world performance can depend on protocol details hard do model.

A more realisitic approach is to \empg{simulate} a client based on the model. 


\section{Node based with fixed neighbor set and add-hoc anonymity (static)}



\section{Churn model extension with bootstrapping and offline detection (dynamic)}
\section{(??) Extension of the model by example}
